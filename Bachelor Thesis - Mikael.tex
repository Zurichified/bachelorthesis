\documentclass{seal_thesis}

\thesisType{Bachelor Thesis}
\date{\today}
\title{Effects of source code properties on variability of software microbenchmarks written in Go}

\author{Mikael Basmaci}
\home{Istanbul} % Geburtsort
\country{Turkey}
\legi{15-721-244}
\prof{Prof. Dr. Harald C. Gall}
\assistent{Christoph Laaber}
\email{mikael.basmaci@uzh.ch}
\url{<url if available>}
\begindate{25.03.2019}
\enddate{25.09.2019}

\begin{document}
\maketitle

\frontmatter

\begin{acknowledgements}
	Here comes the acknowledgements.
\end{acknowledgements}

\begin{abstract}
	Here comes the abstract.
\end{abstract}

\begin{zusammenfassung}
	Here comes the summary.
\end{zusammenfassung}

\tableofcontents
\listoffigures
\listoftables
\lstlistoflistings

\mainmatter

\chapter{Introduction}
The importance of testing software systems for performance has grown in recent years.
Software developers can analyze the performance of parts of their software with software
microbenchmarks, which can be defined as unit tests for performance of a software.
Executing microbenchmarks for a defined time period, one can sample an average execution
time of the benchmark, as well as the data points as the results of multiple iterations.
The variability of these results may depend on a lot of factors such as the execution platform,
the hardware the benchmarks are executed on, or even the programming language of the
software itself. It is important to predict the variability of these results to select the stable
benchmarks to be run, because executing performance tests of a software is usually a long,
time consuming process. Moreover, by predicting the variability, the stability of benchmarks can
be estimated. By understanding the root causes for benchmark variability and being able topredict these, we could support developers to write better benchmarks. One of the ways to predict the variability might be through analyzing source code properties of the software. This
can on one hand help the developers identify the cause of slowdowns in a newer version for
example, on the other hand help them understand which source code property affects the
results in which way.

\chapter{Related Work}
Here comes the literature research about benchmark variability. While some papers look at the benchmark variability causes, some of them try to predict the performance regressions based on different versions of a software by analyzing different commits and mining source code.

\chapter{Methodology}
Here comes the methodology we use during this project. This includes the categorization of benchmarks of 228 projects written in Go by looking at the total executions, the mean, coefficient of variation and relative confidence interval width with 95\% and 99\% confidence intervals. Secondly, we introduce the methodology we used to extract source code properties from in the first place chosen benchmarks. Thirdly, comparing the source code properties with the variabilities of benchmarks by using regression models to achieve the results.

\chapter{Results}
In this section, we present the results from this project. Hopefully there will be  significant, interesting and appliable results that contribute to science.

\chapter{Discussion}
In this section, we discuss the results that we obtained and how relevant these are.

\chapter{Threats to the validity}
If needed, there can be this section about the threats to the validity of this bachelor thesis report. Maybe we did something wrong whilst analyzing the data, maybe the variabilities are not realistic?

\chapter{Conclusion}
We finally conclude with what we have done with this project: How we started, which steps we took and which results we achieved.


\backmatter
\bibliographystyle{alpha}
\bibliography{}

\end{document}
